\chapter{Conclusion générale}

\setlength{\footskip}{50pt}

L'imagerie cardiovasculaire par résonance magnétique chez le petit animal est confrontée à de nombreuses problématiques que ce soit en terme de résolution spatiale et temporelle, en contraste, en sensibilité aux mouvements ou en temps d'acquisition. A cause de ces limitations,  l'IRM cardiaque chez la souris était généralement réalisée avec des acquisitions en 2D ce qui résultait en une faible résolution selon la direction de coupe. Or, dans le cadre d'étude préclinique, les paramètres quantitatifs comme la volumétrie cardiaque ou les mesures de vitesses des flux nécessitent des mesures précises. Il est donc important de disposer d'une résolution spatiale élevée dans les trois dimensions qui permettra une visualisation optimale du système cardiovasculaire ainsi que des mesures robustes. 
En revanche l'imagerie 3D soulève d'autres problèmes comme le manque de signal du sang en imagerie cardiaque dû à l'absence d'effet temps-de-vol ainsi que les temps d'acquistions élevés en particulier dans le cas de l'imagerie 4D (3D résolue dans le temps).
C'est dans cette optique que l'échantillonnage des données selon une trajectoire radiale a été exploré en alternative aux stratégies conventionnelles cartésiennes. 
\medskip

La première étape a été de réaliser une analyse des propriétés des trajectoires radiales qui a révélé qu'elles offraient divers avantages comme la possibilité de sous échantillonner l'acquisition ainsi qu'une plus faible sensibilité aux mouvements. D'autres propriétés comme la sensibilité aux artefacts d'off-resonance ou l'utilisation d'un champ de vue circulaire ou sphérique se trouve être désavantageux et limitant pour une utisation des séquences radiales en routine. Au vu de ces points, il n'est pas envisageable que l'imagerie radiale remplace un jour l'imagerie cartésienne de manière systématique, mais il est probable que son utilisation devienne de plus en plus importante pour des applications spécifiques qui bénificieront des avantages de cette trajectoire. Les exemples de champ d'application où l'utilisation de méthodes radiales semblent envisageable sont l'IRM interventionnelle, l'IRM cardiaque, l'angiographie ainsi que l'imagerie des tissus à $T_2/T_2^*$ court.
\medskip

Durant cette thèse. plusieurs méthodes ont été proposées qui permettent une utilisation des méthodes radiales pour de nombreuses applications précliniques et ont été validées à champs magnétiques élevées (4.7T, 7T et 9.4T).
Ces nouvelles méthodes ont permis d'obtenir des avancées par rapport aux méthodes décrites dans la littérature sur plusieurs points.

Premièrement, les méthodes radiales sont très robustes aux artefacts de mouvements grâce au moyennage des données au centre de l'espace de Fourier. Cette propriété est présente que ce soit en imagerie radiale PR ou UTE. Cette propriété a été utilisée dans toutes les applications de cette thèse puisque aucune n'est synchronisée sur la respiration, cependant il est à noter que l'amplitude des mouvements respiratoires est limitée par le positionnement de l'antenne ce qui réduit la présence de ce type d'artefact chez la souris. 
De plus, les séquences radiales UTE sont aussi extrêmement robustes aux artefacts de flux, cela a été exploité pour les trois méthodes d'imagerie cardiaque UTE combinées à l'injection de nanoparticules de fer puisque l'on obtient des images avec un signal sanguin très homogène même durant la phase cardiaque systolique que ce soit dans le coeur, la crosse aortique ou les autres vaisseaux.

Un deuxième point important est le gain en signal. En effet avec les séquences UTE, la diminution du TE permet de limiter la décroissance du signal des tissus et donc une augmentation du SNR. Cela permet aussi d'observer les tissus avec un $T_2^*$ court comme le poumon où les zones fortement inhomogène en terme de champ magnétique, par exemple aux interfaces air/tissus. Mais surtout, il a été montré pour la première fois que l'injection d'agents de contraste à base de nanoparticules de fer permet de générer un contraste positif grâce aux TE très courts qui peuvent être obtenus avec les séquences radiales UTE et ceci à haut champ magnétique (de 4.7 à 9.4T). Cette augmentation de signal dans le système vasculaire permet d'obtenir un bon contraste avec le myocarde dont le rehaussement est plus faible et donc de parfaitement visualiser l'anatomie cardiaque.

Troisièmement, la trajectoire radiale permet d'obtenir des acquisitions flexibles. Cela est généralement exploité dans des applications d'imagerie dynamique après l'injection d'un agent de contraste. Mais dans cette thèse cette propriété est exploitée dans le cadre d'acquisition 3D ciné. En effet, grâce aux trajectoires radiales il est possible d'utiliser des schémas de remplissage de l'espace de Fourier originaux qui permettent de combiner les données acquises lors de la reconstruction de différentes manières. Cette spécificité est exploitée pour augmenter la résolution spatiotemporelle de la méthode d'angiographie dynamique grâce à des trajectoires doublement pseudo-aléatoire. Bien que cela ne soit pas n'ait pas été montré dans cette thèse, cette stratégie peut aussi être utilisée dans le cas de l'imagerie stack-of-star UTE. Le choix du filtre a posteriori permet de modifier la résolution spatiotemporelle ce qui est particulièrement intéressant dans le cas de l'imagerie préclinique où la qualité des images peut être impactée par de nombreux facteurs comme le type d'animaux, le rythme cardiaque etc. La flexibilité des trajectoires radiales a aussi été exploité pour permettre de reconstruire à partir d'un même jeu de donnée des images avec une forte résolution spatiale et des images avec une forte résolution temporelle du coeur que ce soit en imagerie avec une synchronisation sur le rythme cardiaque prospective ou rétrospective.

Enfin, bien qu'en imagerie radiale le nombre de signaux à recueillir pour respecter le critère de Nyquist soit plus important qu'en cartésien, il est possible de ne pas le satisfaire. En effet, les artéfacts obtenues en cas de sous-échantillonnage sont peu gênants pour l'interprétention que ce soit en imagerie 2D (artefacts de "striking") ou en imagerie 3D (artefacts diffus ressemblant à du bruit). Cette propriété permet d'atteindre des facteurs de sous-échantillonnage important. Les facteurs de sous-échantillonnage des séquences utilisées durant cette thèse atteint des valeurs supérieures à 10 dans le cas de l'angiographie dynamique mais est généralement de l'ordre de 5.

La combinaison de toutes ces avancées a permis de réduire le temps d'acquisition des séquences pour toutes les méthodes par rapport aux précédents travaux dans la littérature puisqu'il n'est plus nécessaire d'accumuler le signal autant qu'en imagerie cartésienne pour compenser le manque de signal, les artefacts de mouvements. 
Il est donc maintenant possible d'effectuer l'imagerie du coeur en 4D en 12 minutes avec des résolutions spatiales isotropiques inférieures à (200 $\mu m^3$) ou des acquisitions de 35 minutes pour obtenir des résolutions de (104 $\mu m^3$). 
Ces avancées ont aussi lieu en terme de résolution temporelle puisqu'il est possible d'obtenir des visualisations des mouvements cardiaques avec des périodes inférieures à 4 ms pour des temps d'acquisitions raisonables (< 35 minutes).

\medskip
Comme abordé précédement, Il est possible d'étudier diverses pathologies cardiaques ou vasculaires à l'aide de modèles murins spécifiques. Les méthodes développées dans cette thèse ouvrent de nouvelles possibilités d'étude grâce à l'augmentation des résolutions spatiales et temporelles.

A priori, les deux méthodes d'imagerie cardiaque proposées permettent de couvrir la plupart des pathologies. La méthode d'angiographie 4D UTE permet d'obtenir une très forte résolution temporelle et spatiale du coeur cependant celle-ci nécessite d'avoir un signal ECG suffisant ainsi qu'un rythme cardiaque stable chez l'animal, cette méthode est donc difficilement applicable par exemple dans le cas d'infarctus sévère. Dans ces types de pathologie, la méthode 4D stack-of-star UTE autosynchronisée sur le rythme cardiaque est plus adaptée car elle permet la synchronisation sans l'ECG. De plus, la reconstruction étant rétrospective cette méthode est plus robuste aux modifications du rythme cardiaque qui peuvent être fréquentes dans ce type de pathologie. Cependant ces avantages se font au prix d'une plus faible résolution dû à un espace de Fourier non uniformément rempli. Les deux méthodes sont donc complémentaires pour pouvoir étudier tout type de pathologie.

Les méthodes de mesure de flux que ce soit en angiographie dynamique ou par codage de phase sont elles aussi complémentaires. La méthode d'angiographie se limite aux régions anatomiques permettant d'obtenir un effet temps-de-vol et n'est donc pas disponible sur le coeur où les artères du poumon. Cette méthode est orientée vers la mesure de flux "rapide" en un temps d'acquisition extrêmement restreint permettant d'envisager une mesure des vitesses sur le corps entier. La méthode par codage de phase demande des temps d'acquisitions importants mais permet de visualiser les flux sanguins au sur tout le thorax (coeur et poumons). L'utilisation de facteur Venc faible permet d'obtenir une mesure des flux faibles et il est possible d'obtenir une reconstruction anatomique des images grâce au très bon contraste entre le sang et les tissus adjacent obtenu avec les agents de contraste. De plus, il est possible avec ce type de donnée de calculer des paramètres comme le "Wall Shear Stress" ou des informations de pression.

\medskip
Pour conclure, l'application de ces méthodes chez l'homme nécessite d'adapter ces méthodes aux contraintes spécifiques de l'imagerie clinique.

Les temps d'acquisitions de toutes ces méthodes doivent être réduits pour pouvoir entrer en routine clinique. L'avantage des IRMs cliniques est la présence de nombreux canaux de réception sur les antennes ainsi qu'un signal-sur-bruit plus important, cela permet l'utilisation de stratégie d'imagerie parallèle et donc des gains en terme de temps d'acquisition généralement compris entre 2 et 4. Cependant ces méthodes d'accélérations sont plus complexes à combiner avec les trajectoires radiales et donc requiert un temps de reconstrution plus élevé, en particulier en imagerie radiale 3D.

Un deuxième point à noter est la présence de mouvements respiratoires beaucoup plus amples. Ceux-ci devront être corrigés soit en détectant les projections corrompues et en les recueillant à nouveau. Ou bien, en utilisant la flexibilité des séquences radiales avec des trajectoires pseudo-aléatoire qui permettront de supprimer les données corrompues de manière uniforme dans l'espace de Fourier et d'obtenir des images de qualité suffisante grâce aux propriétés de sous-échantillonnage des séquences radiales.

Dans le cas des séquences nécessitant une injection d'agent de contraste à base de nanoparticule de fer, une validation des produits en usage clinique sera nécessaire car aucun de ces produits n'est actuellement approuvé par les autorités en tant qu'agents de contraste. Depuis quelques années, un intérêt nouveau s'est porté vers le Ferumoxytol qui est un produit injecté à forte dose dans le cas d'anémie car celui-ci dispose de caractéristiques intéressantes en terme de demi-vie $r_1$/$r_2$. Celui-ci a récemment reçu un avertissement "Black Box" dû à des réactions aléargiques parfois fatales chez les patients mais il est à noter que dans le cas d'une utilisation en tant qu'agent de contraste la dose injectée est bien plus faible et injectée à un débit contrôlé ce qui peut limiter les effets observés précédement. Des études utilisants une faible dose de Ferumoxytol ont été réalisées en imagerie pédiatrique sans faire état de complication \cite{han2014four,Nayak:2014aa}.  Quel que soit l'agent de contraste à base de nanoparticules de fer utilisé, la concentration des doses à injecter nécessitera d'être étudier que ce soit en terme de toxicité puisque avec les séquence UTE le signal est rehaussé avec la concentration de l'agent de contraste.

Pour la séquence de 4D flux UTE, son utilisation en combinaison avec l'injection de nanoparticules de fer nécessitera de diminuer la dose puisque le TE de cette séquence sera fortement augmentée dans le cas de séquence nécessitant un encodage des flux avec une faible vitesse. En effet, les gradients sur les imageurs cliniques sont de l'ordre de 70 mT/m alors que sur les imageurs préclinique ceux-ci sont de 660 mT/m, il faudra donc utiliser des gradients bipolaire avec une durée plus longue. Cependant cette méthode peut aussi être utilisée sans injection d'agent de contraste et semble prometteuse pour sa robustesse dans le cas de flux turbulents \cite{Kadbi:2014uq}.

Toutes les méthodes présentées dans cette thèse ont un intérêt potentiel pour des applications cliniques, leurs implémentation et adaption est l'une des perspectives de ces travaux qui devront ensuite être validé par rapport aux méthodes existantes.
