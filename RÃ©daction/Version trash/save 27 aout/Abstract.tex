%% Abstract
\thispagestyle{empty}

\newgeometry{ inner=2cm,
  outer=2cm,
  top=1cm,
  bottom=1cm,
  headsep=1.2cm}
  
\renewcommand{\baselinestretch}{1}
\line(1,0){400}  \\

\noindent
\small{\textbf{Développement méthodologiques basés sur des trajectoires radiales en imagerie cardiovasculaire par résonance magnétique chez le petit animal.}


L'imagerie cardiovasculaire chez le rongeur par RMN est encore aujourd'hui un véritable défi que ce soit au niveau de la résolution spatiale et temporelle à atteindre mais aussi en terme de temps d'acquisition pour des acquisitions en 3D.
Dans cette optique, l'utilisation de trajectoire non cartésienne et en particulier radiale sont une alternative intéressante aux méthodes standardes cartésiennes. En effet ces trajectoires bénéficient d'autres avantages comme leurs faibles sensibilités aux artefacts de mouvements et de flux ainsi que la possibilité de fortement sous-échantillonner les acquisitions sans obtenir d'artefact cohérent. L'objectif de cette thèse a été de développer de nouvelles méthodes utilisant les propriétés des acquisitions radiales pour l'imagerie cardiovasculaire 3D anatomique et fonctionnelle chez le petit animal à hauts champs magnétiques.
Au cours de cette thèse nous avons développés deux méthodes d'imagerie anatomique 3D+t: l'une des méthodes est basé sur une acquisition radiale à temps d'écho ultracourt (UTE) associée à l'injection de nanoparticules de fer permettant de générer à partir d'un même jeu de donnée des images avec une forte résolution spatiale ou temporelle. La seconde méthode basé sur une acquisition hybride UTE-cartésienne permet de se synchroniser sur le rythme cardiaque a posteriori lors de la reconstruction à partir des données RMN acquises et a permis d'obtenir des images avec une forte résolution sur des modèles d'animaux dont les signaux électriques de conduction dans le coeur sont perturbés.
Deux méthodes de mesure de flux ont aussi été développées: L'une basée sur une acquisition ciné temps-de-vol 4D et permettant de visualiser l'avancée des flux dans l'arbre vasculaire après la saturation du volume d'imagerie. L'utilisation de trajectoire radiale associée à une répartition avec deux angles d'or a permis de réduire le temps d'acquisition à 5 minutes. La seconde est basé sur le principe de contraste de phase, la combinaison d'une séquence radiale à temps d'écho ultracourt et de l'injection de nanoparticules de fer a permis de générer des cartes de vitesses sur l'entiéreté du coeur.

\noindent
\textbf{Mots clés : } IRM du petit animal, trajectoire radiale, mesure de flux, angiographie
}

\line(1,0){400} \\

\noindent
\small{\textbf{Développement méthodologiques basés sur des trajectoires radiales en imagerie cardiovasculaire par résonance magnétique chez le petit animal.}


L'imagerie cardiovasculaire chez le rongeur par RMN est encore aujourd'hui un véritable défi que ce soit au niveau de la résolution spatiale et temporelle à atteindre mais aussi en terme de temps d'acquisition pour des acquisitions en 3D.
Dans cette optique, l'utilisation de trajectoire non cartésienne et en particulier radiale sont une alternative intéressante aux méthodes standardes cartésiennes. En effet ces trajectoires bénéficient d'autres avantages comme leurs faibles sensibilités aux artefacts de mouvements et de flux ainsi que la possibilité de fortement sous-échantillonner les acquisitions sans obtenir d'artefact cohérent. L'objectif de cette thèse a été de développer de nouvelles méthodes utilisant les propriétés des acquisitions radiales pour l'imagerie cardiovasculaire 3D anatomique et fonctionnelle chez le petit animal à hauts champs magnétiques.
Au cours de cette thèse nous avons développés deux méthodes d'imagerie anatomique 3D+t: l'une des méthodes est basé sur une acquisition radiale à temps d'écho ultracourt (UTE) associée à l'injection de nanoparticules de fer permettant de générer à partir d'un même jeu de donnée des images avec une forte résolution spatiale ou temporelle. La seconde méthode basé sur une acquisition hybride UTE-cartésienne permet de se synchroniser sur le rythme cardiaque a posteriori lors de la reconstruction à partir des données RMN acquises et a permis d'obtenir des images avec une forte résolution sur des modèles d'animaux dont les signaux électriques de conduction dans le coeur sont perturbés.
Deux méthodes de mesure de flux ont aussi été développées: L'une basée sur une acquisition ciné temps-de-vol 4D et permettant de visualiser l'avancée des flux dans l'arbre vasculaire après la saturation du volume d'imagerie. L'utilisation de trajectoire radiale associée à une répartition avec deux angles d'or a permis de réduire le temps d'acquisition à 5 minutes. La seconde est basé sur le principe de contraste de phase, la combinaison d'une séquence radiale à temps d'écho ultracourt et de l'injection de nanoparticules de fer a permis de générer des cartes de vitesses sur l'entiéreté du coeur.

\noindent
\textbf{Mots clés : } IRM du petit animal, trajectoire radiale, mesure de flux, angiographie
}



\line(1,0){400}