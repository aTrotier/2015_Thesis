\chapter{Introduction}
\setlength{\footskip}{50pt}
\label{Chap1}

Durant les trente dernières années, le développement de méthodes permettant de modifier facilement le génotype ainsi que l'amélioration de la micro-chirurgie ont permis de mimer de nombreuses pathologies humaines chez l'animal ou bien d'en modifier la physiologie. Ces avancées ont joué un rôle important dans la recherche biomédicale qui a vu une forte augmentation de l'utilisation de modèles animaux, en particulier de souris et de rat. En effet, ceux-ci sont régulièrement utilisés pour leurs faibles coûts d'entretien, leurs cycles de reproduction courts, leurs disponibilités et leurs relatives facilités de transport. En parallèle de ces avancées, de nombreuses méthodes permettant d'étudier ces modèles ont été découvertes, développées, perfectionnées ou adaptées à une utilisation sur le petit animal. Développer l'imagerie est crucial afin de pouvoir phénotyper et suivre ces modèles. 
\medbreak

Plusieurs modalités sont disponibles pour l'imagerie du petit animal comme la tomographie par émission de positons (TEP) ou monophotonique (TEMP), la tomodensitométrie à rayons X, l'imagerie optique, l'imagerie ultrasonore et l'imagerie par résonance magnétique (IRM). Chaque technique présente des avantages et des inconvénients que ce soit en termes de résolution spatiale, de résolution temporelle, de sensibilité, de spécificité, etc. De par leurs particularités inhérantes aux principes physiques, ces modalités sont donc complémentaires. Certaines seront à favoriser en fonction de l'application et du type d'information à recueillir.
\medbreak
%
Dans le cas de l'imagerie du système cardiovasculaire de la souris \textit{in vivo} plusieurs spécificités sont à noter. La modalité doit être la moins invasive possible, suffisamment sensible, offrir une résolution spatiale élevée et des contrastes élevés modulables et enfin plusieurs types d'informations doivent être recueillies :
\begin{enumerate}
 \item Des informations anatomiques sur la forme des vaisseaux, du cœur etc.
 \item Des informations fonctionnelles sur la contraction du coeur, la dilatation des artères ou bien les vitesses des flux.
\end{enumerate}
Seules l'imagerie par ultrasons et l'IRM peuvent répondre à ces besoins. Cependant l'IRM permet d'obtenir une visualisation en profondeur des tissus contrairement aux ultrasons dont la profondeur est limitée (de 5 à 15 mm) par la résolution spatiale que l'on souhaite obtenir. De plus, l'imagerie 3D disponible en IRM rend l'acquisition très peu opérateur-dépendante alors que la plupart des systèmes ultrasonores disponibles ne permettent que des acquisitions 2D et donc des mesures peu reproductibles. L'IRM, malgré son coût élevé, apparaît donc comme un candidat idéal puisqu'elle répond à la plupart des exigences de l'imagerie du petit animal.

\medbreak
Le principe de l'IRM a vu le jour en 1973 dans une publication de Paul Lauterbur \cite{lauterbur1973image} après 30 ans de recherches sur la résonance magnétique nucléaire (RMN) qui consiste à mesurer le signal produit par un échantillon placé en présence d'un champ magnétique statique et qui est ensuite excité par une onde de radiofréquence oscillant à la fréquence particulière de résonance du noyau que l'on souhaite imager. Depuis, de nombreuses avancées techniques ont fait de l'IRM un outil indispensable dans le domaine de l'imagerie bio-médicale. Aujourd'hui, les imageurs par résonance magnétique utilisés en clinique ont généralement des champs statiques de 1,5T ou 3T qui permettent d'obtenir de manière non invasive des images avec différents contrastes selon les paramètres utilisés. Ce très bon contraste permet d'obtenir une très bonne visualisation des structures anatomiques. La percée de l'IRM repose fortement sur sa capacité à obtenir des informations sur une large étendue d'autres paramètres allant de la densité de proton, de la diffusion, des vitesses de flux et de la température jusqu'à des paramètres plus complexes comme la distribution du sang, l'activité cérébrale ou bien l'orientation des faisceaux de fibres nerveuses.

\medbreak
D'un autre côté, l'IRM fait face à deux principales limitations qui sont accrues lors du passage de l'imagerie clinique à l'imagerie préclinique. La première est la faible sensibilité de cette modalité qui pose problème dans le cas de l'imagerie de la souris puisqu'il y a un rapport d'environ 1/3000 avec la masse de l'homme. Les résolutions à atteindre pour pouvoir observer les régions anatomiques sont donc bien plus importantes ($< 200 \mu m$ isotrope), ce qui génère une considérable diminution du rapport signal-sur-bruit des images. Pour compenser cette diminution, la tendance a été d'augmenter le champ statique des imageurs précliniques afin d'accroître la polarisation. Ainsi, les imageurs précliniques ont généralement des valeurs de champ statique supérieures à 4,7T. Cependant, cette augmentation provoque d'autres problèmes comme une diminution des valeurs $T_1$ des tissus, une modification des effets des agents de contraste, une baisse de l'homogénéité du champ magnétique et une augmentation du dépôt d'énergie lors de l'excitation radiofréquence. Une autre alternative pour compenser la faible sensibilité est l'accumulation du signal mais cette méthode allonge fortement les temps d'acquisition car l'augmentation du rapport signal-sur-bruit est proportionnelle à la racine carrée du nombre d'accumulation.
La deuxième différence importante entre l'humain et la souris est la fréquence des mouvements physiologiques cardiaque et respiratoire. Celles-ci sont bien plus importantes, par exemple, le fréquence cardiaque d'une souris non anesthésiée est de 500 à 600 battements par minutes (bpm) contre 50 à 90 chez l'Homme. La résolution temporelle d'acquisition des images doit donc être augmentée pour permettre d'observer le cœur durant différentes phases du cycle cardiaque.

\medbreak
Pour ces deux principales raisons, l'IRM cardiovasculaire chez la souris est un véritable défi et il est nécessaire de développer des séquences d'acquisitions adaptées au petit animal que ce soit en terme de gain en rapport signal-sur-bruit, de réduction du temps d'acquisition ou de sensibilité aux artefacts de mouvements. L'utilisation de séquences avec des trajectoires alternatives semble être une voie prometteuse et en particulier les trajectoires non cartésiennes qui permettent de progresser sur chacune de ces limitations. L'idée d'utiliser des trajectoires non cartésiennes d'échantillonnage du signal n'est pas nouvelle. Lauterbur utilisait déjà l'une d'entre elles dans son article de 1973 \cite{lauterbur1973image}, cependant cette méthode fut rapidement remplacée par les méthodes cartésiennes grâce à leur faible sensibilité aux erreurs instrumentales des premiers systèmes d'imagerie par résonance magnétique. Dans ces travaux de thèse c'est cette stratégie connue sous le nom d'imagerie radiale qui sera développée car elle présente des propriétés intéressantes par rapport à l'imagerie cartésienne qui viennent de la géométrie particulière de recueil des données.

\medbreak
L'objectif de cette thèse est de développer de nouvelles méthodes 3D d'imagerie cardiovasculaire chez le petit animal basées sur des trajectoires radiales d'échantillonnage du signal. L'aspect 3D des séquences n'est pas anodin puisqu'il permet d'augmenter la résolution dans la direction de coupe, de gagner en rapport signal-sur-bruit par rapport à des séquences en 2D. De plus, dans le cas de résolutions pratiquement isotropes, le positionnement des coupes d'imagerie est facilité car il est possible de reconstruire des coupes \textit{a posteriori} selon d'autres plans. Cependant les acquisitions 3D requièrent généralement un temps d'acquisition plus long. Une partie du travail dans cette thèse a consisté à réduire ces durées par rapport à celles que l'on peut trouver actuellement dans la littérature.

Dans cette optique, nous avons développé quatre méthodes d'IRM cardiovasculaire 4D (3D résolue dans le temps). 
Deux méthodes permettant d'obtenir des informations anatomiques et fonctionnelles et deux autres méthodes permettant de visualiser et de quantifier la vitesse des flux sanguins. 
\medbreak


Cette thèse est organisée de la manière suivante : Le chapitre 2 présente le principe de l'imagerie radiale et évoque ses propriétés principales ainsi que la reconstruction particulière de ce type d'acquisition. Le chapitre 3 présente l'amélioration d'une séquence développée au sein du laboratoire en 2006 permettant de visualiser l'avancée des flux sanguins. L'implémentation d'une méthode radiale de type projection-reconstruction couplée à une répartition des trajectoires selon une méthode doublement pseudo-aléatoire a permis de fortement réduire les temps d'acquisitions tout en augmentant la résolution spatiale. 
Les chapitres 4 à 6 présentent de nouvelles méthodes basées sur l'injection d'agents de contraste à base de nanoparticules de fer couplée à l'utilisation de séquences radiales à temps d'écho ultracourts. Cette combinaison permet de rehausser le signal du sang et d'obtenir un contraste positif entre le sang et le myocarde durant une période compatible avec le temps d'acquisition des séquences. Le chapitre 4 présente une séquence 4D (3D résolue dans le temps) anatomique synchronisée sur le rythme cardiaque grâce à des électrodes ECG. La stratégie développée permet de reconstruire, à partir d'un même jeu de données, des images avec une forte résolution spatiale et des images avec une forte résolution temporelle. Le chapitre 5 s'attarde sur une méthode d'imagerie anatomique 4D auto-synchronisée sur le rythme cardiaque permettant l'imagerie de modèle animaux dont les signaux électriques de conduction dans le coeur sont perturbés et qui limitent l'utilisation d'électrodes pour la synchronisation sur le rythme cardiaque. Dans le chapitre 6, une méthode de quantification de la vitesse des flux 4D est détaillée, basée sur un module d'encodage des vitesses selon les 3 directions de l'espace positionné avant la lecture du signal radial. Cette utilisation de la trajectoire radiale permet de limiter les artefacts provoqués par les flux turbulents lors de la contraction du coeur.
Le chapitre 7 récapitulera les principales avancées de ce travail et donnera un aperçu sur les futurs travaux cliniques et précliniques.





