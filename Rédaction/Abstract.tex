%% Abstract
\thispagestyle{empty}

\newgeometry{ inner=2cm,
  outer=2cm,
  top=1cm,
  bottom=1cm,
  headsep=1.2cm}
  
\renewcommand{\baselinestretch}{1}
\line(1,0){400}  \\

\noindent
\small{\textbf{Nouvelles stratégies d'acquisitions non cartésiennes pour l'IRM cardiovasculaire du petit animal.}

L’imagerie cardiovasculaire par RMN est encore aujourd’hui un véritable défi. La difficulté réside dans la nécessité d’acquérir des images avec de fortes résolutions spatiale et temporelle en un temps limité, et dans certains cas sur des zones en mouvement. Alors que la plupart des images sont acquises avec des trajectoires cartésiennes, notre choix s'est porté sur l’utilisation de trajectoires 3D radiales comme alternative. En effet, celles-ci bénéficient de nombreux avantages comme leur faible sensibilité aux artefacts de mouvements et de flux ainsi que la possibilité de fortement sous-échantillonner les acquisitions. Ainsi, l’objectif de cette thèse a été de développer de nouvelles méthodes utilisant les propriétés des acquisitions radiales pour l'imagerie cardiovasculaire 3D anatomique et fonctionnelle chez le petit animal à hauts champs magnétiques.

Tout d'abord, une méthode de mesure des flux sanguins en 3D a été mise au point, basée sur le phénomène de temps-de-vol. L’utilisation de trajectoires radiales a permis de réduire fortement les temps d’acquisition tout en améliorant les résolutions spatiale et temporelle des images par rapport aux méthodes cartésiennes.

Ensuite, en combinant l’utilisation de nanoparticules de Fer qui possèdent une rémanence vasculaire importante avec des séquences radiales à temps d’écho ultracourt, nous avons montré que l’acquisition d’images anatomiques cardiaques et vasculaires très haute résolution pouvait être réalisée de manière prospective ou bien retrospective grâce à l’ajout d’un écho-navigateur dans la séquence permettant l’auto-synchronisation cardiaque.

Enfin, cette même méthode a été employée pour réaliser l’imagerie de flux 4D sur l’entièreté du système cardio-pulmonaire de l’animal.

Les séquences développées lors de ce travail et les résultats obtenus en imagerie anatomique et fonctionnelle montrent l’intérêt et la robustesse des méthodes non cartésiennes en imagerie préclinique. Elles peuvent ouvrir la voie à de nouvelles stratégies en imagerie clinique.

%L'imagerie cardiovasculaire chez le rongeur par RMN est encore aujourd'hui un véritable défi que ce soit au niveau de la résolution spatiale et temporelle à atteindre mais aussi en terme de temps d'acquisition pour des acquisitions en 3D.
%Dans cette optique, l'utilisation de trajectoire non cartésienne et en particulier radiale sont une alternative intéressante aux méthodes standardes cartésiennes. En effet ces trajectoires bénéficient d'autres avantages comme leurs faibles sensibilités aux artefacts de mouvements et de flux ainsi que la possibilité de fortement sous-échantillonner les acquisitions sans obtenir d'artefact cohérent. L'objectif de cette thèse a été de développer de nouvelles méthodes utilisant les propriétés des acquisitions radiales pour l'imagerie cardiovasculaire 3D anatomique et fonctionnelle chez le petit animal à hauts champs magnétiques.
%Au cours de cette thèse nous avons développés deux méthodes d'imagerie anatomique 3D+t: l'une des méthodes est basé sur une acquisition radiale à temps d'écho ultracourt (UTE) associée à l'injection de nanoparticules de fer permettant de générer à partir d'un même jeu de donnée des images avec une forte résolution spatiale ou temporelle. La seconde méthode basé sur une acquisition hybride UTE-cartésienne permet de se synchroniser sur le rythme cardiaque a posteriori lors de la reconstruction à partir des données RMN acquises et a permis d'obtenir des images avec une forte résolution sur des modèles d'animaux dont les signaux électriques de conduction dans le coeur sont perturbés.
%Deux méthodes de mesure de flux ont aussi été développées: L'une basée sur une acquisition ciné temps-de-vol 4D et permettant de visualiser l'avancée des flux dans l'arbre vasculaire après la saturation du volume d'imagerie. L'utilisation de trajectoire radiale associée à une répartition avec deux angles d'or a permis de réduire le temps d'acquisition à 5 minutes. La seconde est basé sur le principe de contraste de phase, la combinaison d'une séquence radiale à temps d'écho ultracourt et de l'injection de nanoparticules de fer a permis de générer des cartes de vitesses sur l'entiéreté du coeur.

\noindent
\textbf{Mots clés : } IRM préclinique, 3D+t, trajectoire radiale, cardiovasculaire, mesure de flux.
}

\line(1,0){400} \\

\noindent
\small{\textbf{New strategies of non-cartesian acquisitions for cardiovascular small animal MRI.}

Cardiovascular imaging using NMR is still a real challenge. The difficulty relies on the need to acquire images with high temporal and spatial resolutions, in a limited acquisition time and in some cases of moving areas. While most images are acquired with cartesian trajectories, the use of 3D radial trajectories was explored as an alternative. Indeed, they benefit from various advantages like their low sensitivity to flow and motion artefacts as well as the opportunity to highly undersample acquisitions. Thus, the aim of this thesis was to develop new acquisition strategies using radial trajectory properties for 3D cardiovascular anatomical and functional imaging in small animals at high magnetic fields.

First, a method for measuring blood flow in 3D was developped, based on a time-of-flight effect. The use of radial trajectories allowed to highly reduce acquisition times while increasing spatial and temporal resolutions compared to cartesian acquisitions.

Then, combining the injection of iron nanoparticles which have a long vascular remanence with ultrashort echot time radial acquisitions, we showed that anatomical cardiac images with a high spatial resolution could be obtained prospectively or restrospectively by adding a navigator echo in the sequence in order to synchronize the reconstruction to the cardiac cycle.

Finally, this method was used to perform 4D flow imaging on the entire cardiopulmonary system of the animals.

The sequences developed during this work and the results obtained in anatomical and functional imaging show the interest and the robustness of non cartesian methods in preclinical imaging. They paves the way to the development of new strategies in clinical imaging.



\noindent
\textbf{Keywords : } Preclinical MRI, 3D+t, radial trajectories, cardiovascular, flow measurement.
}



\line(1,0){400}